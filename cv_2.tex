%%%%%%%%%%%%%%%%%%%%%%%%%%%%%%%%%%%%%%%%%
% Medium Length Graduate Curriculum Vitae
% LaTeX Template
% Version 1.1 (9/12/12)
%
% This template has been downloaded from:
% http://www.LaTeXTemplates.com
%
% Original author:
% Rensselaer Polytechnic Institute (http://www.rpi.edu/dept/arc/training/latex/resumes/)
%
% Important note:
% This template requires the res.cls file to be in the same directory as the
% .tex file. The res.cls file provides the resume style used for structuring the
% document.
%
%%%%%%%%%%%%%%%%%%%%%%%%%%%%%%%%%%%%%%%%%

%----------------------------------------------------------------------------------------
%	PACKAGES AND OTHER DOCUMENT CONFIGURATIONS
%----------------------------------------------------------------------------------------

\documentclass[10pt]{res} % Use the res.cls style, the font size can be changed to 11pt or 12pt here

\usepackage{helvet} % Default font is the helvetica postscript font
% \usepackage{newcent} % To change the default font to the new century schoolbook postscript font uncomment this line and comment the one above
\usepackage{geometry}
\usepackage{hanging}

\geometry{
 a4paper,
 total={170mm,257mm},
%  left=10mm,
%  right=1mm,
%  bottom=1mm,
 top=8mm,
 }
\setlength{\textwidth}{6in} % Text width of the document

\begin{document}

%----------------------------------------------------------------------------------------
%	NAME AND ADDRESS SECTION
%----------------------------------------------------------------------------------------

\moveleft.5\hoffset\centerline{\large\bf Joshua K. Carr, BSc} % Your name at the top
 
\moveleft\hoffset\vbox{\hrule width\resumewidth height 1pt}\smallskip % Horizontal line after name; adjust line thickness by changing the '1pt'
 
\moveleft.5\hoffset\centerline{260 rue Mutchmore, apt 302} % Your address
\moveleft.5\hoffset\centerline{Gatineau, QC, J8Y 3V3}
\moveleft.5\hoffset\centerline{(343) 550 4687}
\moveleft.5\hoffset\centerline{josh.carr@carleton.ca}

%----------------------------------------------------------------------------------------

\begin{resume}

%----------------------------------------------------------------------------------------
%	OBJECTIVE SECTION
%----------------------------------------------------------------------------------------
 
% \section{OBJECTIVE}  

% To obtain an entry-level internship in a technology-oriented company to leverage my passion for human-computer interaction and emerging technologies. I am primarily interested utilizing new and future technologies to discover new possibilities and re-imagine the ways in which we engage with technology.

%----------------------------------------------------------------------------------------
%	Technology SKILLS SECTION
%----------------------------------------------------------------------------------------

\section{SKILLS} 

{\sl Technologies:} HTML/CSS/JavaScript, Python 3, R, SPSS, GNU/Linux, Bash, \LaTeX, MS Office. 
\vspace{2mm}\\
{\sl Research:} Advanced multivariate statistics, experimental design, usability/UX analysis, data collection. 
\vspace{2mm}\\
{\sl Communication:} Writing for refereed publications, peer-review, presenting research findings, facilitating lectures and seminars, creating data visualizations. 
\vspace{2mm}\\
{\sl Other:} Leadership, team-orientation, detail-orientation, technology literacy, problem-solving, healthcare technologies, working with vulnerable populations.


%----------------------------------------------------------------------------------------
\section{EDUCATION}
{\sl Master of Applied Science, Human-Computer Interaction} \hfill September 2018-Present\newline
Carleton University, Ottawa, ON
\begin{itemize}
    \item Anticipated completion in January 2020.
\end{itemize}{}

{\sl Bachelor of Science (Honours), Psychology} \hfill September 2011 - April 2015\newline
Trent University, Peterborough, ON


%----------------------------------------------------------------------------------------

\section{EXPERIENCE}

{\sl Junior Business Analyst} \hfill May - August 2019 \\
Service Canada, Gatineau, QC
\begin{itemize}\itemsep -2pt % Reduce space between items
    \item Internship program through the Collaborative Learning of Usability Experiences (CLUE) program at Carleton University.
    \item Quantitative data analysis, data visualization, and communication to inform business decisions.
    \item Qualitative research methods, especially Journey Mapping, applied to accessibility problems in vulnerable populations.
\end{itemize} 

% {\sl Graduate Student, MASc Human-Computer Interaction} \hfill September 2018 - Present \\
% Carleton University, Ottawa, ON
% \begin{itemize}\itemsep -2pt % Reduce space between items
%     \item Thesis project applying brain-computer interface technologies to computer security and user authentication. Focus on usability and human factors.
%     \item Supervised by Dr. Robert Biddle
% \end{itemize} 

{\sl Polysomnograph Technologist} \hfill May 2017 - August 2018 \\
Trent Regional Sleep Clinic, Peterborough, ON
\begin{itemize}\itemsep -2pt % Reduce space between items
    \item Worked with patients in a clinical setting. Setup and overnight monitoring of patients during diagnostic sleep tests. Patients included vulnerable populations such as children, elderly, disabled and cognitively delayed individuals.
    \item Worked with a small interdisciplinary team in a highly collaborative environment.
    \item Responded to emergency medical situations, initiated treatment for sleep-disorders and hospital transfers.
    \item Developed familiarity with biometric sensors and healthcare technologies (e.g. electronic medical records).
\end{itemize} 

% \newpage
% \section{EXPERIENCE}
{\sl Research Assistant} \hfill September 2015 - April 2017 \\
Trent University, Peterborough, ON 
\begin{itemize} \itemsep -2pt % Reduce space between items
    \item Designed and executed empirical research studies in the domain of molecular neurobiology.
    \item Conducted statistical analyses and hypothesis testing on quantitative data.
    \item Communicated experimental findings via research papers, oral-visual presentations, and posters.
\end{itemize}

{\sl Teaching Assistant} \hfill September 2015 - April 2017 \\
Trent University, Peterborough, ON 
\begin{itemize} \itemsep -2pt % Reduce space between items
    \item Assisted with teaching and administration of various undergraduate psychology courses.
    \item Presented lectures to students, facilitated small-group seminars, met with students one-on-one to assist with course material. 
\end{itemize}



% {\sl Undergraduate Student, BSc Psychology} \hfill September 2011 - 2015\\
% Trent University, Peterborough, ON
% \begin{itemize} \itemsep -2pt % Reduce space between items
%     \item Completed an Honours thesis under supervision of Dr. Hugo Lehmann focusing on the neural representation of spatial memories in rodents. Thesis was later adapted into an article and published in a refereed journal.
%     \item Worked as an NSERC-funded research assistant during the summers of 2014 and 2015.
%     \item Attended and presented research findings at the international Society for Neuroscience conference in 2015 and 2016. 
% \end{itemize} 


 %----------------------------------------------------------------------------------------
%	AWARDS AND ACHIEVEMENTS
%---------------------------------------------------------------------------------------- 
% \newpage
\section{AWARDS AND ACHIEVEMENTS}

\vspace{5mm}


{\sl \textbf{Refereed Publications:}}\\
Carr, J.K., Fournier, N.M., \& Lehmann, H. (2015). Increased task demand during spatial memory testing recruits the anterior cingulate cortex. Learning and Memory 23:9. doi:  10.1101/lm.042366.116


Kalinina, A., Maletta, T., Carr, J.K., Lehmann, H., \& Fournier, N.M. (2019). Spatial exploration induced expression of immediate early genes Fos and Zif268 in adult-born neurons is reduced after pentylenetetrazole kindling. Brain Research Bulletin. doi: 10.1016/j.brainresbull.2019.07.003

\vspace{5mm}
{\sl \textbf{Conference Presentations:}}\\
Carr, J.K., Kalinina, A., Lehmann, H., \& Fournier, N.M. (2016). The effects of amygdala kindling on hippocampal neurogenesis and pattern separation. Program No. 461.11/LLL14. San Diego, CA: Society for Neuroscience Annual Meeting, 2016.

Kalinina, A., Carr, J.K, Turner, H., Lehmann, H., \& Fournier, N.M. (2016). Effect of chronic seizures on the functional integration of adult born neurons. Program No. 594.26/K3. San Diego, CA: Society for Neuroscience Annual Meeting, 2016.

Carr, J.K., Fournier, N.M., Lehmann, H. (2015). Increased task demand during a spatial memory retention test recruits the anterior cingulate cortex. Program No. 83.07/X13. Chicago, IL: Society for Neuroscience Annual Meeting, 2015.

\vspace{5mm}

{\sl \textbf{Scholarships:}}\\
NSERC Undergraduate Student Research Award (2014 and 2015)


\end{resume}
\end{document}